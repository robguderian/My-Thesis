\chapter{Conclusion and Future Work}
\label{ch:conclusion}

In both simulation and testbed deployments, HCCP has increased message throughput while
having very little negative effects on network lifespan. HCCP provides 
a robust and tuneable network that provides many options to elect motes
that have certain properties as clusterheads more frequently. 
Depending on the configuration, HCCP appears to be able to provide network lifespans 
as long as LEACH while having almost double the message throughput, as shown in both 
simulations and physical deployments.

While HCCP is more complex than LEACH, it has a more robust design, and provides
infrastructure for disseminating information across the network. HCCP relies on 
the network to have schedules in tight synchronization, which is a
requirement in all WSN protocols, the novelty that HCCP adds is to 
use this highly coupled time to describe how good a
mote would be at being a clusterhead. The Goodness Delay
utilizes time that would otherwise not be used to optimize 
which motes will be clusterheads. 

HCCP is designed to be an easily tuneable protocol, giving 
the network designer tools to make the best network possible. 
An example of tuning the network is 
changing the length of the Roundtable Discussion time. This time
is allotted to share any network-specific information that 
doesn't fit anywhere else into the network. HCCP provides some suggestions
as to what to share during this time, but is intended to be customized 
for every individual network, or even left out if desired.

Heterogeneity in a WSN can be a powerful thing, and HCCP uses this
heterogeneity to elect better clusterheads. The definition of `better' is
up to the network designer to choose. HCCP allows the election to focus 
on motes that have qualities that a network designer feels are important 
to the network. This thesis provides insight into what heterogeneous 
factors have to offer to the network, how the affect the election, message
throughput and network/mote lifespans.

Overall, heterogeneity, whether the differences are small  or large, can be used 
to generate better functioning WSNs. Choosing motes that are better for the task of clusterhead 
will prevent lost messages with no cost to the lifespan of the network. The gains of 
using heterogeneity that is already inherent in every WSN is free, having few negative 
effects. Having the heterogeneity of a network self-assessed will ensure that 
as a network degrades over time that the best possible motes at that time will
be doing tasks that they are well-suited for, providing gains to the network from the time
it is deployed to the time the last mote ceases to function.

% future work

Looking forward, HCCP could be simplified for further energy savings. For instance,
the Clusterhead Candidacy phase could likely be merged with the Clusterhead Announcement phase,
as there is little difference to the network lifespan or message throughput when the phases are merged. 
Since removing the
Clusterhead Candidacy would make HCCP simpler and would not 
have any negative affects on the network, it would probably be a good idea. Further
research could done on how HCCP could surpass LEACH in terms of network lifespan.

A better routing protocol should be 
incorporated into the design of HCCP. More research could be done 
to see which routing algorithms would be the best with HCCP. The routing
algorithm should also take advantage of the heterogeneity of the network, routing 
messages through stronger motes.
HCCP was designed with this in mind, and is easily changed to use any routing algorithm. 
Further, if a mobile sink is a consideration, then using a routing algorithm that works efficiently with
networks that have a mobile sink, such as MobiRoute~\cite{mobileSinkRouting}, is also possible.

Querying motes in the network from the sink was not a focus of HCCP. If a network administrator 
wants to request data from a mote, there is no set method of sending the messages to any mote in the network
but the sink. To add querying to HCCP, a different routing method would need to be used, as only the  
sink's position is advertised while using beacon routing. A routing protocol that has a table
of all the mote's last known positions would need to be used. Once routing is set up, the 
querying framework would need to be added. This feature could be added into the 
Roundtable Discussion time, or into a new phase that is dedicated to mote querying.

HCCP was not designed with any inherent security measures. If messages are solely to be routed to the sink, 
simple RSA encryption~\cite{rsa}
could be used to encrypt the messages that are being sent to the sink. Other security issues such as
an errant mote that is dying continually becoming a clusterhead could be solved by blacklisting motes
that are continuously clusterheads.

To create a WSN with fewer collisions in the network,
it is possible to use Frequency Division Multiple access (FDMA)~\cite{fdma} to use
different radio channels for different clusters. 
Clusterheads could announce the channel along with the TDMA schedule. 
Using multiple radio channels during the TDMA clustermote reporting run time would remove many of the collisions
that happen during the TDMA run time.


Radio power could also be adjusted to suit the size of the network. Adjusting the radio power to the lowest transmission
power saves mote energy, and prevents motes from flooding the network with unecessary messages from motes. Also, smaller transmission 
ranges make clusters smaller. Changing the radio power effectively changes the cluster size, since few motes will be in 
range. 
Lin et al.~\cite{atpc}, Zheng et al.~\cite{adaptive2010}, Xiao and Yu~\cite{efficientRadioPower} and others have 
developed efficient methods of adjusting the radio power to enable motes to talk 
to a limited number of neighbouring motes. Messages to communicate radio power could be added
to the HCCP's Roundtable Discussion, or piggybacked on announcement messages, such as clusterhead announcements.


\section{Future Goals of WSNs}


Smart Dust~\cite{smartdust-2, SmartDust-NC}, is one of the potential futures for WSNs. As circuits, processors
and motes get smaller and smaller there may come a point where motes are no larger than the size of dust.
These motes would have near negligible unit cost and be deployed in the thousands. Smart
Dust would communicate wirelessly, monitoring some phenomena and glean energy from the environment
to stay powered. Smart Dust would need to be deployed with a dense network that would need to be a multi-hop 
network, as each device would have minimal battery power and minimal transmission power. Any protocols designed
would have to be scalable to accommodate large networks of Smart Dust. HCCP has been designed to scale, but 
smart dust might be a larger scale than HCCP was designed for, though HCCP provides interesting ways
to use the differences in the dust well.

As devices get smaller and more feature-rich, the eventual outcome might be
that all devices are networked and can communicate with each other. When all devices
can communicate with each other, they would create `an internet of things', a term 
used frequently by Adam Dunkels~\cite{vasseur10interconnecting}, who is the 
creator of Contiki OS~\cite{dunkels04contiki, contikiOS} and a key member of the WSN community.
Dunkels et al.~\cite{dunkels11adhoc} believe that devices will communicate with each other using a
lightweight version of IPv6 called 6LoWPAN IPv6~\cite{jhui10ipv6}. A subset of the larger IPv6 protocol~\cite{ipv6} 
is an obvious choice, as it can provide unique addresses
for $3.4 \times 10^{38}$ devices, which would provide 
%$4.8 \times 10^{28}$ 
48~000~000~000~000~000~000~000~000~000 addresses
for every one of the 7 billion 
humans on Earth. This means that every single device in the Internet of Things could have 
it's own address, and be able to communicate with any other device in the internet of things. 
This would make the entire Internet a massive scale WSN of sorts, with many heterogeneous 
devices all connected using the common language of IPv6. HCCP uses IPv6 addressing, allowing 
networks running HCCP to possibly be part of the internet of things.


